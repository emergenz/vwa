% !TeX root = document.tex
% TODO: Soll die Zusammenfassung im Inhaltsverzeichnis angeführt werden?
% \chapter*{Abstract} verhindert den Eintrag im Inhaltsverzeichnis.
% \addchap{Abstract}
\pdfbookmark[0]{Abstract}{abstract}
\chapter*{Abstract}
Diese Arbeit befasst sich mit allgemeiner künstlicher Intelligenz, also künstlicher Intelligenz mit domänenübergreifender Lernkapazität, und mit der Anpassung maschineller Werte an die menschlichen bei einem solchen System. Sie zeigt die Auswirkungen einer allgemeinen künstlichen Intelligenz auf und legt Ansätze zur Lösung des Anpassungsproblems dar. Konkret wird auf die Idee der KI-Sicherheit durch KI-Debatten eingegangen. Bei dieser handelt es sich um ein Nullsummen-Debattierspiel, bei dem zwei KIs auf eine Fragestellung antworten, abwechselnd Argumente liefern und dabei versuchen, das jeweils letzte Argument des Gegners zu entkräften. Im Schlussteil der Arbeit wird Verbesserungspotential an der Idee der KI-Debatten angeführt und eine internationale Institution für AKI-Forschung als Maßnahme vorgeschlagen, um die Entwicklung einer angepassten AKI zu gewährleisten.

%Als allgemeine künstliche Intelligenz (AKI) bezeichnet man ein technisch fortgeschrittenes System, dessen Lernkapazität nicht auf einzelne Domänen begrenzt ist, sondern als \emph{allgemein} bezeichnet werden kann. Der Meinung führender KI-Experten nach ist es sehr wahrscheinlich, dass die KI-Forschung bis 2075 zu einer AKI führt. Für ein solches System sind Menschen nur eine Ansammlung an Atomen, die auch für das Erreichen seiner Ziele eingesetzt werden können. Eine AKI kann Menschen schaden, ohne dass sie Werte besitzt, die dies explizit fordern. Um dies zu verhindern, müssen die Werte der AKI an die Werte der Menschheit angepasst werden. Diese Arbeit legt einen Ansatz einer solchen Werteanpassung dar: die der KI-Sicherheit durch KI-Debatten.

%Es handelt sich um ein Nullsummen-Debattierspiel, bei dem zwei KIs auf eine Fragestellung antworten, abwechselnd Argumente liefern und dabei versuchen, das jeweils letzte Argument des Gegners zu entkräften. Das Spiel endet, wenn der menschliche Begutachter genug Informationen hat, um einen Fehler in der Argumentationslinie eines Spielers auszumachen. So kann ein Begutachter auch Verhalten beurteilen, das für ihn sonst zu komplex oder unverständlich wäre.

%Bei der Entwicklung einer AKI sind Regulationen seitens internationaler Institutionen notwendig, da eine unangepasste AKI leistungsfähiger wäre als eine angepasste. Durch Implementierung einer \emph{Windfall-Klausel} können auch andere negative Auswirkungen einer AKI wie Arbeitslosigkeit oder eine Machtkonzentration ausgeglichen werden.
%%% Local Variables:
%%% mode: latex
%%% TeX-master: "document"
%%% End:
