% !TeX root = document.tex
\chapter{Einleitung}

Ich möchte diese Arbeit mit einer Analogie beginnen.

Nehmen wir an, Sie hätten eine Pille. Sie wissen, dass diese Pille entweder alle Krankheiten, Ungleichheiten, Schmerzen - kurz gesagt alles \quotes{Schlechte} - auf der Welt beseitigen, oder aber die gesamte Menschheit auslöschen könnte. Würden Sie die Pille nehmen? Wahrscheinlich nicht. Nun stellen wir uns vor, nicht Sie hätten die Pille, sondern sie befände sich irgendwo auf unserem Planeten und \emph{jeder} könnte sie finden. Wären Sie besorgt? Wahrscheinlich schon.

Dann sollten Sie dies schon heute sein, denn eine solche Pille wird es geben - ihr Name: allgemeine künstliche Intelligenz.

Was rechtfertigt eine so technovolatile Haltung wie diese?

\zit[30:51--31:07]{noauthor_eliezer_nodate}{There are all sorts of extreme forces coming onto the game board that were not there before. To expect them to all fail or exactly cancel out for the purpose of making the outcome normal would be one heck of a coincidence.}

Jede technologische Neuentdeckung bedeutet in erster Linie Veränderung. Die Erfindungen der letzten Jahrhunderte hatten mehrheitlich positive Auswirkungen zur Folge, sonst wäre unser Lebensstandard heute nicht der höchste in der Menschheitsgeschichte.\vgl[22--23]{easterlin_worldwide_2000} So ermutigend das auch klingt, so dürfen wir nicht einfach nach dem Trend der Vergangenheit in die Zukunft extrapolieren, sondern müssen - so Richard A. Easterlin - versuchen, die Kräfte zu verstehen, die für den Anstieg der Lebensqualität verantwortlich sind. \vgl[23]{easterlin_worldwide_2000} Was eine allgemeine künstliche Intelligenz betrifft, müssen wir sie nicht nur verstehen, sondern auch lenken können, um das Wohlbefinden der Spezies Mensch nicht zu gefährden, sondern zu bestärken.

