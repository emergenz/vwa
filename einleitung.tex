% !TeX root = document.tex
\chapter{Einleitung}

Ich möchte diese Arbeit mit einem Gedankenexperiment beginnen.

Es existiere ein System, dass durch ein quantitativ und qualitativ höheres Intelligenzniveau in der Lage ist, Ziele zu erreichen, die die Menschheit ohne eine solches System nicht erreichen könnte. Der Eigentümer einer Büroklammernfabrik sei im Besitz eines solchen Systems und gebe diesem das Ziel, so viele Büroklammern wie möglich herzustellen. Am Anfang beginnt das System, die Arbeitsabläufe in der Fabrik zu automatisieren. Nach einiger Zeit durchlebt es eine Intelligenzexplosion, optimiert sich selbst immer weiter und beginnt, Menschen zu töten, um aus ihnen Büroklammern herzustellen und hört damit nicht auf, bis das gesamte Universum nur noch aus Büroklammern besteht. \vgl[123-124]{bostrom_superintelligence:_2014}

Ein solches System mit einer allgemeinen künstlichen Intelligenz könnte beim Erreichen der ihnen vorgegebenen Ziele nebenbei die gesamte Menschheit auslöschen.

Obiges Szenario wäre die Folge einer allgemeinen künstlichen Intelligenz, die nicht genau das macht, was der Mensch von ihr will. Die Maschine kennt die Werte der Menschheit nicht. Sie weiß nicht, dass sie keinem Menschen Schaden zufügen darf, dass ihr Operator seinen Gewinn maximieren will oder dass die Erhaltung der Umwelt von höherer Priorität ist als das Herstellen von Büroklammern. Diese Arbeit beschäftigt sich mit der Anpassung eines Systems an menschliche Werte -- also mit der maschinellen Werteanpassung --, um ein Szenario wie das oben genannte zu vermeiden. Dabei werden die folgenden beiden Leitfragen beantwortet:

\begin{enumerate}
\item Welche Folgen kann es nach Schaffung einer allgemeinen künstlichen Intelligenz geben?
\item Kann man eine allgemeine künstliche Intelligenz so programmieren, dass der Mensch immer die Kontrolle über sie behält?
\end{enumerate}

Die Beantwortung dieser Fragen soll mit Hilfe von Literatur sowie wissenschaftlichen Arbeiten erfolgen.

Das erste Kapitel dient zur Begriffserklärung, im zweiten werden die Auswirkungen einer allgemeinen künstlichen Intelligenz genannt und im dritten werden Lösungsansätze für das Problem der maschinellen Werteanpassung dargelegt.
%Was rechtfertigt diese technovolatile Haltung?

%\zit[30:51--31:07]{noauthor_eliezer_nodate}{There are all sorts of extreme forces coming onto the game board that were not there before. To expect them to all fail or exactly cancel out for the purpose of making the outcome normal would be one heck of a coincidence.}

%Jede technologische Neuentdeckung bedeutet in erster Linie Veränderung. Die Erfindungen der letzten Jahrhunderte hatten mehrheitlich positive Auswirkungen zur Folge, sonst wäre unser Lebensstandard heute nicht der höchste in der Menschheitsgeschichte.\vgl[22--23]{easterlin_worldwide_2000} So ermutigend das auch klingt, so dürfen wir nicht einfach nach dem Trend der Vergangenheit in die Zukunft extrapolieren, sondern müssen -- so \citeauthor{easterlin_worldwide_2000} -- versuchen, die Kräfte zu verstehen, die für den Anstieg der Lebensqualität verantwortlich sind. \vgl[23]{easterlin_worldwide_2000} Was eine allgemeine künstliche Intelligenz betrifft, müssen wir sie nicht nur verstehen, sondern auch lenken können, um das Wohlbefinden der Spezies Mensch nicht zu gefährden, sondern zu stärken.


%%% Local Variables:
%%% mode: latex
%%% TeX-master: "document"
%%% End:
