% !TeX root = document.tex
% 20180308T1028 Leonard Michlmayr

%% Einige Filter für die Einträge im Literaturverzeichnis
\defbibfilter{online}{( type=online or subtype=online )}
\defbibfilter{interview}{type=interview or subtype=interview}
\defbibfilter{onlinetext}{( type=online or subtype=online and not ( type=video
  or type=audio ) )}
\defbibfilter{offline}{not ( type=online or subtype=online )}
\defbibfilter{print}{not ( type=online or subtype=online or type=video or
  type=audio or type=interview or subtype=interview )}
\defbibfilter{offlinevideo}{type=video and not subtype=online}
\defbibfilter{offlineaudio}{type=audio and not subtype=online}
\defbibfilter{nurAusSekundaerliteratur}{category=quotee and not category=primary}
\defbibfilter{nichtNurAusSekundaerliteratur}{%
  category=quoter or category=primary or category=nocited}

% TODO: die Untergliederung des Literaturverzeichnisses den eigenen
% Bedürfnissen anpassen.
\printbibheading[heading=bibintoc,title=Literaturverzeichnis]\label{Lit}
\printshorthands[heading=subbibintoc]
\printbibliography[heading=subbibintoc,title={Print-Quellen},category=inbib,filter=print,filter=nichtNurAusSekundaerliteratur]
\printbibliography[heading=subbibintoc,title={Audio-Quellen},category=inbib,filter=offlineaudio,filter=nichtNurAusSekundaerliteratur]
\printbibliography[heading=subbibintoc,title={Video-Quellen},category=inbib,filter=offlinevideo,filter=nichtNurAusSekundaerliteratur]
\printbibliography[heading=subbibintoc,title={Internet-Quellen},category=inbib,filter=online,filter=nichtNurAusSekundaerliteratur]
\printbibliography[heading=subbibintoc,title={Sekundärzitate},category=inbib,filter=nurAusSekundaerliteratur]
\printbibliography[heading=subbibintoc,title={Interviews},category=inbib,filter=interview]
