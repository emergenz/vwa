% !TeX root = document.tex
\chapter{Allgemeine künstliche Intelligenz}
\section{Definition von Intelligenz}
Seit Jahrhunderten versuchen Wissenschaftler und Laien gleichermaßen eine Definition für den Intelligenzbegriff zu finden. Da bis heute keine Defintion ihre Vollständig- oder Richtigkeit beweisen konnte, wird in dieser Arbeit der Einfachheit halber versucht, den Begriff durch Beobachtungen zu erklären, wie Eliezer Yudkowsky in dem Podcast \quotes{AI: Racing Toward the Brink} vorschlägt.
\begin{enumerate}
\item Menschen waren auf dem Mond.
\item Mäuse waren nicht auf dem Mond.
\end{enumerate}
Yudkowsky wählt dieses Beispiel um zu demonstrieren, dass Menschen auch Orte erreichen, wofür die natürliche Selektion sie nicht vorbereitet hat. Daraus könne geschlossen werden, dass Menschen \emph{intelligenter} als Mäuse sind, weil sie \emph{domänenübergreifend} arbeiten können. Deshalb sei das \emph{domänenübergreifende} Erlernen neuer Fähigkeiten ein zentraler Teil des Intelligenzbegriffs. \vgl[06:01--09:49]{EliezerPodcast}
\section{Definition von künstlicher Intelligenz}
\zit{kaplan_siri_2019}{Artificial intelligence (AI)–—defined as a system’s ability to correctly interpret external data, to learn from such data, and to use those learnings to achieve specific goals and tasks through flexible adaptation}

Laut angeführter Definiton muss eine künstliche Intelligenz nicht nur Daten richtig interpretieren, sondern auch die dadurch gewonnen Erkenntnisse mittels \emph{dynamischer Anpassung} zur Erreichung bestimmter Ziele benützen können.

Diese Definition enthält die Idee des \emph{domänenübergreifenden} Lernens im Gegensatz zum oben beschriebenen Ansatz zur Intelligenzerklärung nicht, was laut Experten jedoch nicht an einer unvollständigen Definition liegt, sondern vielmehr daran, dass wir den Begriff der KI in einer Art gebrauchen, wofür er nicht vorgesehen war. Um Missverständnisse zu vermeiden, wird für KI wie sie heutzutage bereits in Benutzung ist der Begriff schwache KI (engl. \emph{weak AI} oder \emph{narrow AI}) verwendet. \vgl[18--19]{bostrom_superintelligence:_2014} Dieser beschreibt eine \emph{domänenspezifische} KI.
\section{Definition von allgemeiner künstlicher Intelligenz}
Als allgemeine künstliche Intelligenz bezeichnet man ein technisch fortgeschrittenes System, dessen Lernkapazität nicht auf einzelne Domänen begrenzt ist, sondern als \emph{allgemein} bezeichnet werden kann. \vgl[1]{goertzel_advances_2007}
\section{Wann wird es sie geben?}
\section{Die These der Intelligenzexplosion}

