\message{ !name(document.tex)}% Vorlage für die VWA. Version 20190202 (c) Leonard Michlmayr

% TODO: Wähle den für deine Arbeit die passenden Optionen!
\documentclass[DLS,
	inreferencehack,
	ohneVgl=false,
	ohneS=false,
	scauthor,
	rundeauslassung=false,
	bookstyle=false,
	widowlines=3,
	titlepage=DLS2017,
	listof=nochaptergap,
	doppelpunkt=false,
	postnotedoppelpunkt=false,
	zitierstil=klassisch]{vwa}

\newcommand{\quotes}[1]{``#1''}
%% Trick texlipse to use biber instead of bibtex
\iffalse
\usepackage[error,backend=biber]{biblatex}
\fi
%%

% TODO: lade Zusatzpakete
\usepackage{textcomp}
\usepackage[output-decimal-marker={,}]{siunitx}

% TODO: Eigene Quellendatenbank laden.
\addbibresource{meineBibliothek.bib}
\addbibresource{quellen.bib}

% TODO: Lege das Verzeichnis fest, wo Bilder liegen sollen
\graphicspath{{img/}} 

% Eigenen Namen und Geschlecht wählen.
\Autor{Franz Srambical}
% Klasse einsetzen
\Klasse{8C}
% Betreuungslehrer oder Betreuungslehrerin einsetzen:
\Betreuer{Prof. Mag.\ 
            Kurt Rauch \& Mag. Leonard Michlmayr}
% Das "`Thema"' einsetzen
\Thema{Maschinelle Werteanpassung bei einer hypothetischen allgemeinen künstlichen Intelligenz}

% TODO: erst bei der letzten Version das Abgabedatum anführen
% \Abgabedatum{\today}

\begin{document}

\message{ !name(kapitel1.tex) !offset(-51) }
 % !TeX root = document.tex
\chapter{Allgemeine künstliche Intelligenz}
\section{Definition von Intelligenz}
Seit Jahrhunderten versuchen Wissenschaftler und Laien gleichermaßen eine Definition für den Intelligenzbegriff zu finden. Da bis heute keine Defintion ihre Vollständig- oder Richtigkeit beweisen konnte, wird in dieser Arbeit der Einfachheit halber versucht, den Begriff durch Beobachtungen zu erklären, wie Eliezer Yudkowsky in dem Podcast \quotes{AI: Racing Toward the Brink} vorschlägt.
\begin{enumerate}
\item Menschen waren auf dem Mond.
\item Mäuse waren nicht auf dem Mond.
\end{enumerate}
Yudkowsky wählt dieses Beispiel um zu demonstrieren, dass Menschen auch Orte erreichen, wofür die natürliche Selektion sie nicht vorbereitet hat. Daraus könne geschlossen werden, dass Menschen \emph{intelligenter} als Mäuse sind, weil sie \emph{domänenübergreifend} arbeiten können. Deshalb sei das \emph{domänenübergreifende} Erlernen neuer Fähigkeiten ein zentraler Teil des Intelligenzbegriffs. \vgl[06:01--09:49]{EliezerPodcast}
\section{Definition von künstlicher Intelligenz}
\zit{kaplan_siri_2019}{Artificial intelligence (AI)–—defined as a system’s ability to correctly interpret external data, to learn from such data, and to use those learnings to achieve specific goals and tasks through flexible adaptation}

Laut angeführter Definiton muss eine künstliche Intelligenz nicht nur Daten richtig interpretieren, sondern auch die dadurch gewonnen Erkenntnisse mittels \emph{dynamischer Anpassung} zur Erreichung bestimmter Ziele benützen können.

Diese Definition enthält die Idee des \emph{domänenübergreifenden} Lernens im Gegensatz zum oben beschriebenen Ansatz zur Intelligenzerklärung nicht, was laut Experten jedoch nicht an einer unvollständigen Definition liegt, sondern vielmehr daran, dass wir den Begriff der KI in einer Art gebrauchen, wofür er nicht vorgesehen war. Um Missverständnisse zu vermeiden, wird für KI wie sie heutzutage bereits in Benutzung ist der Begriff schwache KI (engl. \emph{weak AI} oder \emph{narrow AI}) verwendet. \vgl[18--19]{bostrom_superintelligence:_2014} Dieser beschreibt eine \emph{domänenspezifische} KI.
\section{Definition von allgemeiner künstlicher Intelligenz}
Als allgemeine künstliche Intelligenz bezeichnet man ein technisch fortgeschrittenes System, dessen Lernkapazität nicht auf einzelne Domänen begrenzt ist, sondern als \emph{allgemein} bezeichnet werden kann. \vgl[1]{goertzel_advances_2007}
\section{Werte einer allgemeinen künstlichen Intelligenz}
Es ist essenziell, dass ein so fortgeschrittenes System wie eine AKI die Werte der Menschheit teilt, um ungewollten Nebenwirkungen wie der in der Einleitung genannten Auslöschung der Menschheit durch unpräzises Definieren ihrer Ziele mit größtmöglicher Sicherheit vorzubeugen. Dabei geht es nicht darum, eine - wie Ray Kurzweil es in einem Artikel ausdrückt - antropomorphe Maschine \vgl{yudkowsky_what_2001}, also ein System mit menschenähnlichen Eigenschaften zu entwickeln, sondern dafür zu sorgen, dass ein solches System einen gleichermaßen positiven Effekt auf die gesamte Menschheit hat.
\section{Wann wird es sie geben?}
Eine Befragung durch die KI-Wissenschaftler V. C. Müller und N. Bostrom kam zu dem Ergebnis, dass KI-Experten dem Erreichen einer allgemeinen künstlichen Intelligenz in den Jahren 2040 bis 2050 eine Wahrscheinlichkeit von über 50, und dem Erreichen bis 2075 eine Wahrscheinlichkeit von 90 Prozent zuordnen. \vgl[566]{muller_future_2016} Es ist also - sollten sich die Expertenmeinungen als richtig herausstellen - davon auszugehen, dass eine AKI bereits in diesem Jahrhundert zur Realität und bereits für die jetzige Generation mehr als nur relevant sein wird. 
\section{Die These der Intelligenzexplosion}

%%% Local Variables:
%%% mode: latex
%%% TeX-master: "document"
%%% End:

\message{ !name(document.tex) !offset(48) }

\end{document}

%%% Local Variables:
%%% mode: latex
%%% TeX-master: t
%%% End:
