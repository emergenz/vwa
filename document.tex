% Vorlage für die VWA. Version 20190202 (c) Leonard Michlmayr

% TODO: Wähle den für deine Arbeit die passenden Optionen!
\documentclass[DLS,
	inreferencehack,
	ohneVgl=false,
	ohneS=false,
	scauthor,
	rundeauslassung=false,
	bookstyle=false,
	widowlines=3,
	titlepage=DLS2017,
	listof=nochaptergap,
	doppelpunkt=false,
	postnotedoppelpunkt=false,
	zitierstil=klassisch]{vwa}

\newcommand{\quotes}[1]{``#1''}
%% Trick texlipse to use biber instead of bibtex
\iffalse
\usepackage[error,backend=biber]{biblatex}
\fi
%%

% TODO: lade Zusatzpakete
\usepackage{textcomp}
\usepackage[output-decimal-marker={,}]{siunitx}

% TODO: Eigene Quellendatenbank laden.
\addbibresource{meineBibliothek.bib}
\addbibresource{quellen.bib}

% TODO: Lege das Verzeichnis fest, wo Bilder liegen sollen
\graphicspath{{img/}} 

% Eigenen Namen und Geschlecht wählen.
\Autor{Franz Srambical}
% Klasse einsetzen
\Klasse{8C}
% Betreuungslehrer oder Betreuungslehrerin einsetzen:
\Betreuerin{Prof. Mag.\ 
            Kurt Rauch}
% Das "`Thema"' einsetzen
\Thema{Maschinelle Werteanpassung bei einer hypothetischen allgemeinen künstlichen Intelligenz}

% TODO: erst bei der letzten Version das Abgabedatum anführen
% \Abgabedatum{\today}

\begin{document}
% Am Anfang keine Seitennummern
\frontmatter

% PDF-Lesezeichen für die Titelseite
\pdfbookmark[0]{Titelseite}{titlepage}
% Titelseite
\maketitle

% !TeX root = document.tex
% TODO: Soll die Zusammenfassung im Inhaltsverzeichnis angeführt werden?
% \chapter*{Abstract} verhindert den Eintrag im Inhaltsverzeichnis.
% \addchap{Abstract}
\pdfbookmark[0]{Abstract}{abstract}
\chapter*{Abstract}
Der Zusammenfassungstext kommt hier her. Abstract ist kein Vorwort und keine
Einleitung! Hier ist ein Absatz voll sinnlosem Text. Bitte erst nach
diesem Absatz weiterlesen. Hier kommt nichts mehr. Es folgen unterschiedlich lange Wörter.
Die Abruchbirne kringelte ihre Hürde in eine unbekannte Überschwänglichkeit, um
so die Sitzordnung der Fensterscheiben in der unteren Waldkante zu verjubeln.
Niemandem ist absichtlich zu kürzen, wessen Woligkeit hier in abermaligem
Abgesang aufgeschlagen ist. Deswegen soll dieser aber nicht heimreisen, sondern
abermals die Einigkeit des Urwalds in die aufgeregte Höhensonne schlagen.
Wenn nicht der hiesige Erdball des aberwitzigen Ungemachs aufgedrungene
Kröte wäre, entschließe ich mich zu unsachgemäßem Handlungsablauf.
Wiegleich zudem ein weiterer Honigkuchen ausbricht.

Hier ist ein Absatz voll sinnlosem Text. Bitte erst nach
diesem Absatz weiterlesen. Hier kommt nichts mehr. Es folgen unterschiedlich lange Wörter.
Die Abruchbirne kringelte ihre Hürde in eine unbekannte Überschwänglichkeit, um
so die Sitzordnung der Fensterscheiben in der unteren Waldkante zu verjubeln.
Niemandem ist absichtlich zu kürzen, wessen Woligkeit hier in abermaligem
Abgesang aufgeschlagen ist. Deswegen soll dieser aber nicht heimreisen, sondern
abermals die Einigkeit des Urwalds in die aufgeregte Höhensonne schlagen.
Wenn nicht der hiesige Erdball des aberwitzigen Ungemachs aufgedrungene
Kröte wäre, entschließe ich mich zu unsachgemäßem Handlungsablauf.
Wiegleich zudem ein weiterer Honigkuchen ausbricht.

Hier ist ein Absatz voll sinnlosem Text. Bitte erst nach
diesem Absatz weiterlesen. Hier kommt nichts mehr. Es folgen unterschiedlich lange Wörter.
Die Abruchbirne kringelte ihre Hürde in eine unbekannte Überschwänglichkeit, um
so die Sitzordnung der Fensterscheiben in der unteren Waldkante zu verjubeln.
Niemandem ist absichtlich zu kürzen, wessen Woligkeit hier in abermaligem
Abgesang aufgeschlagen ist. Deswegen soll dieser aber nicht heimreisen, sondern
abermals die Einigkeit des Urwalds in die aufgeregte Höhensonne schlagen.
Wenn nicht der hiesige Erdball des aberwitzigen Ungemachs aufgedrungene
Kröte wäre, entschließe ich mich zu unsachgemäßem Handlungsablauf.
Wiegleich zudem ein weiterer Honigkuchen ausbricht.

% !TeX root = document.tex
% TODO: Soll das Vorwort im Inhaltsverzeichnis genannt werden?
% Mit \chapter*{Vorwort} wird eine Nennung im Inhaltsverzeichnis verhindert.
% \addchap{Vorwort}
\pdfbookmark[0]{Vorwort}{vorwort}
\chapter*{Vorwort}
Das Vorwort ist optional: d.\,h.\@ man muss kein Vorwort schreiben! Wer will,
kann das in dieser Form tun. Am Ende sollten Ort, Datum und der Name des Autors
des Vorworts angegeben werden. \vgl{Vorwort}


\begin{flushleft}
Wien am \today
\end{flushleft}
\begin{flushright}
\makeatletter\@AutorIn\makeatother
\end{flushright}


% Das Inhaltsverzeichnis soll ein PDF-Lesezeichen aber keinen Eintrag im
% Inhaltsverzeichnis haben.
\cleardoublepage\pdfbookmark[0]{\contentsname}{toc}
% Inhaltsverzeichnis
\tableofcontents

% Hier geht es los.
\mainmatter
% TODO: füge hier deine Kapitel ein!
 % !TeX root = document.tex
\chapter{Allgemeine künstliche Intelligenz}
\section{Definition von Intelligenz} \label{Intelligenzbegriff}
Seit Jahrhunderten versuchen Wissenschaftler und Laien gleichermaßen eine Definition für den Intelligenzbegriff zu finden. Da bis heute keine Defintion ihre Vollständigkeit oder Richtigkeit beweisen konnte, wird in dieser Arbeit der Einfachheit halber versucht, den Begriff durch Beobachtungen zu erklären, wie \citeauthor{EliezerPodcast} in dem Podcast \enquote{AI: Racing Toward the Brink} vorschlägt. \vgl[07:30-09:45]{EliezerPodcast}
\begin{enumerate}
\item Menschen waren auf dem Mond.
\item Mäuse waren nicht auf dem Mond.
\end{enumerate}
Yudkowsky wählt dieses Beispiel, um zwei Thesen zu belegen:

Menschen sind \emph{intelligenter} als Mäuse, weil sie \emph{domänenübergreifend} arbeiten können. Damit sei das \emph{domänenübergreifende} Erlernen neuer Fähigkeiten ein zentraler Teil des Intelligenzbegriffs.


Die natürliche Selektion ist neben der menschlichen Lernfähigkeit eine der wenigen Vorgänge, die zu einer \emph{domänenübergreifenden} Leistungsoptimierung führt, das oben genannte Beispiel belegt jedoch, dass die Menschheit auch Orte erreichen kann, wofür die natürliche Selektion sie nicht vorbereitet hat. Dies und die Tatsache, dass die Evolution Millionen Jahre benötigte, um aus dem Homo Sapien den Homo Erectus zu formen \vgl[508]{grzimek_grzimeks_1979}, während der Mensch mit seinen Entdeckungen und Erfindungen in wenigen Jahrhunderten zur dominantesten Spezies der Erde geworden ist, zeigt, dass der Mensch der schnellere und effizientere Optimierer ist. \emph{Effizienz} ist also ein weiterer Teilaspekt der Intelligenz.\vgl[9]{yudkowsky_intelligence_2013}

\section{Künstliche Intelligenz}
\zit[15]{kaplan_siri_2019}{Artificial intelligence (AI)---defined as a system’s ability to correctly interpret external data, to learn from such data, and to use those learnings to achieve specific goals and tasks through flexible adaptation}

Laut angeführter Definiton muss eine künstliche Intelligenz nicht nur Daten richtig interpretieren, sondern auch die dadurch gewonnen Erkenntnisse mittels \emph{dynamischer Anpassung} zur Erreichung bestimmter Ziele benützen können.

Diese Definition enthält im Gegensatz zum oben beschriebenen Ansatz zur Intelligenzerklärung die Idee des \emph{domänenübergreifenden} Lernens nicht, was laut Experten jedoch nicht an einer unvollständigen Definition liegt, sondern vielmehr daran, dass wir den Begriff der künstlichen Intelligenz (KI) in einer Art gebrauchen, für die er nicht vorgesehen war. Um Missverständnisse zu vermeiden, wird für KI wie sie heutzutage bereits in Benutzung ist der Begriff schwache KI (engl. \emph{weak AI} oder \emph{narrow AI}) verwendet. \vgl[18--19]{bostrom_superintelligence:_2014} Dieser beschreibt eine \emph{domänenspezifische} KI.

\section{Allgemeine künstliche Intelligenz}
Als allgemeine künstliche Intelligenz (AKI; auch \emph{starke KI} genannt; engl. \emph{strong AI} oder \emph{general AI}) bezeichnet man ein technisch fortgeschrittenes System, dessen Lernkapazität nicht auf einzelne Domänen begrenzt ist, sondern als \emph{allgemein} bezeichnet werden kann. \vgl[1]{goertzel_advances_2007}

\section{Werte einer allgemeinen künstlichen Intelligenz}  \label{Werte}
\zit[01:51--01:57]{paul_current_2019}{The goal is to build AI systems that are trying to do what you want them to do}

Der \emph{Instrumental Convergence Thesis} \vgl[9-10]{omohundro_basic_2008} nach gibt es bestimmte Ressourcen, die für eine AKI beim Erreichen der ihnen vorgegebenen Ziele in den meisten Fällen behilflich sind. Dazu gehören unter anderem Materie oder Energie, eine AKI wird jedoch auch Quellcodeveränderungen, die zu einem potenziellen Erschweren ihrer Zielerfüllung führen könnten, zu stoppen versuchen. Sie kann also Menschen schaden, ohne dass sie Werte besitzt, die dies explizit fordern. Für ein rein rational denkendes System sind Menschen nichts als eine Ansammlung von Atomen, die auch für das Erreichen seiner Ziele eingesetzt werden können.\vgl[14]{yudkowsky_intelligence_2013}

Ein fortgeschrittenes System wie eine AKI muss ihre Ziele daher auf der Basis von Werten verfolgen, von denen die Menschheit als Gesamtes profitiert, um ungewollten Nebenwirkungen wie der in der Einleitung genannten Auslöschung der Menschheit durch unpräzises Definieren ihrer Ziele mit größtmöglicher Sicherheit vorzubeugen.

Der Ansatz eine \emph{antropomorphe} Maschine, also ein System mit menschenähnlichen Eigenschaften, zu entwickeln, ist bedenklich. Während einige menschliche Werte und Eigenschaften implementiert werden müssen, um mögliche Dissonanzen zwischen der AKI und der Menschheit zu vermeiden, dürfen andere menschliche Eigenschaften nicht übernommen werden. Ansonsten werden Vorurteile ohne rationalem Grundsatz in das System aufgenommen, was zu systematischer Diskriminierung führt, sodass eine AKI beim Erreichen ihrer Ziele beispielsweise Frauen oder Afrikaner benachteiligt oder Asiaten automatisch als intelligenter einstuft.\vgl{yudkowsky_what_2001}

Menschliche Werte in einer Programmiersprache nachzubilden ist nach der \emph{Complexity of Value Thesis} aufwendig, da sie - selbst in idealisierter Form - eine hohe algorithmische Komplexität vorweisen. Daher muss eine AKI komplexe Informationen gespeichert haben, damit sie die ihr vorgegebenen Ziele auf eine menschengewollte Weise erfüllen kann. Dabei reichen auch keine vereinfachten Zielstellungen wie \quotes{Menschen glücklich machen} \vgl[13-14]{yudkowsky_intelligence_2013}, denn es gibt keinen \quotes{Geist im System}, der diese abstrakte Zielsetzung ohne Weiteres versteht.

\citeauthor{hibbard_super-intelligent_2002} beschreibt in seinem Buch \enquote{Super-Intelligent Machines} eine Möglichkeit, Maschinen das abstrakte Gefühl der Freude zu erklären. Dabei lernt eine hypothetische KI durch einen riesigen Datensatz, bei welchen Gesichtsausdrücken, Stimmeigenschaften und Körperhaltungen ein Mensch glücklich ist.\vgl[115]{hibbard_super-intelligent_2002} Yudkowsky ist der Meinung, dass dies keinesfalls eine Lösung für das Problem der exakten Zielsetzung ist und führt Hibbards Gedankenexperiment fort. Falls diese KI nun ein Bild von einem winzigen, molekularen Smiley-Gesicht sieht, so besteht die Möglichkeit, dass die KI dies als Glücklichsein interpretiert und das Universum in eine einzige Ansammlung von winzigen, molekularen Smiley-Gesichtern umzuwandeln versucht, um den höchstmöglichen Zustand des Glücklichseins zu erreichen. \vgl[3]{yudkowsky_complex_2011}

\section{Wann wird es sie geben?}
Eine Befragung durch die KI-Wissenschaftler V. C. Müller und N. Bostrom kam zu dem Ergebnis, dass KI-Experten dem Erreichen einer AKI in den Jahren 2040 bis 2050 eine Wahrscheinlichkeit von über 50 und dem Erreichen bis 2075 eine Wahrscheinlichkeit von 90 Prozent zuordnen. \vgl[566]{muller_future_2016} Es ist also - sollten sich die Expertenmeinungen als richtig herausstellen - davon auszugehen, dass eine AKI bereits in diesem Jahrhundert zur Realität und bereits für die jetzige Generation relevant sein wird. Kritiker dieser Meinung weisen darauf hin, dass es ähnliche Schätzungen bereits seit den Siebzigerjahren gibt und sie sich immer wieder als falsch herausgestellt haben. \citeauthor{allen_paul_2011} behaupten, es bräuchte noch einige wissenschaftliche Durchbrüche, um eine AKI noch in diesem Jahrhundert zu erreichen. \vgl{allen_paul_2011} Auch die Möglichkeit ihrer Entwicklung ist nicht unumstritten, jedoch gibt es keine Beweise, die darauf hindeuten, dass eine solche Entwicklung unmöglich ist. KI-Sicherheit betrifft aber auch schwache KIs wie sie schon existieren. Es muss alsbald eine Möglichkeit gefunden werden, das Verhalten einer KI an die Werte der Menschheit anzupassen. Eine mögliche AKI würde die Folgen einer \enquote{unangepassten} künstlichen Intelligenz nur verstärken.

\section{Die These der Intelligenzexplosion}
Eine AKI werde - unabhängig von ihren Zielen - Selbstoptimierung hinsichtlich ihrer Intelligenz anstreben, weil sie dadurch ihre Ziele schneller und effizienter erreichen könne. Sobald die erste KI programmiert werden würde, die qualitativ bessere - also noch intelligentere - KIs programmieren könnte, käme es zu einem Kreislauf der kognitiven Leistungssteigerung. Die KI der Tochtergeneration könnte nun als verbesserter KI-Designer noch bessere KIs programmieren. Anders als bei biologischer Intelligenz kann eine KI bei Verfügbarkeit entsprechender Hardware einfach kopiert werden. Eine Gruppe von KIs hätte dann gemeinsam quantitativ und qualitativ höhere kognitive Fähigkeiten, ähnlich einer Schwarmintelligenz. Dieser hypothetische Kreislauf ist die Grundlage der These der Intelligenzexplosion. Nach ihr wird ab einer bestimmten Schwelle die Leistungssteigerung mit jeder KI-Iteration größer, was zu einer \emph{Superintelligenz} führt, die der Menschheit kognitiv deutlich überlegen ist. (Der Intelligenzbegriff wird in dieser Arbeit anhand der Fähigkeit zur Zielerreichung definiert, siehe Kapitel \ref{Intelligenzbegriff}) \vgl[13]{muehlhauser_intelligence_2012}

Einige Informatiker, unter ihnen \citeauthor{lanier_who_2013}, behaupten, dass sich eine Technologie nicht ohne fortlaufenden Input verbessern könne. \vgl[299]{lanier_who_2013} Diese These wurde jedoch empirisch widerlegt. \citeauthor{silver_mastering_2017} haben einen Algorithmus entwickelt, der ohne Vorwissen und ohne jegliche Beispieldaten das Spiel \emph{Go} von Grund auf gelernt hat. Die KI -- AlphaGo Zero ihr Name -- spielt nach einer Trainingszeit von drei Stunden auf dem Niveau eines Anfängers und ist nach drei Tagen besser als der beste menschliche Spieler. Nach 40 Tagen ist sie der beste Go-Spieler der Welt und stärker als jeder andere Go-Computer. \vgl{silver_mastering_2017}

%%% Local Variables:
%%% mode: latex
%%% TeX-master: "document"
%%% End:

% !TeX root = document.tex
\chapter{Probleme einer allgemeinen künstlichen Intelligenz}
\section{Fehlerhafte Vorstellungen einer KI-Katastrophe}
\subsection{Bösartige KI}
\quotes{Ghost in the Machine (complex value systems)}
\subsection{KI, die ein Bewusstsein erlangt}
\subsection{Roboter als Auslöser einer Katastrophe}
\section{Gesamtmenschheitlicher Konsens über gemeinsame Werte}
KOMMENTAR: Reflective Equilibrium; Ideal advisor Theory
\section{\quotes{Gute} und \quotes{schlechte} menschliche Werte}
\section{Wertekodierung in einer Programmiersprache}
\subsection{Statische Wertekodierung}
\subsection{Dynamisch-maschinelle Werteanpassung}
\section{Biases}
\subsection{Verzerrung in der Risikoeinschätzung}
KOMMENTAR: Auch Zeitpunkt einer AKI
\subsection{Verzerrung in der Werteformulierung}
\subsection{Verzerrung in der Kodierung}
Nutzenfunktion (eng. \emph{utility function})
\section{Sichere und vertrauenswürdige KI}
\vgl{yudkowsky_intelligence_2013}
\section{KI-Ethik}

%%% Local Variables:
%%% mode: latex
%%% TeX-master: "document"
%%% End:

%\chapter{Testbeispiele}
\iffalse
\zit[\bibstring{confer}][]{Scherz}{Ein wörtliches Zitat, das mit einem Punkt endet.}
\zit[\bibstring{confer}][]{Scherz}{Ein wörtliches Zitat, das mit einem Fragezeichen endet?}
\zit[\bibstring{confer}][]{Scherz}{Ein wörtliches Zitat, das mit einem Punkt endet}.
\zit[\bibstring{confer}][]{Scherz}{Ein wörtliches Zitat, das mit einem Fragezeichen endet}?
\zit[\bibstring{confer}][]{Scherz}[.]{Ein wörtliches Zitat, das mit einem Punkt endet}
\zit[\bibstring{confer}][]{Scherz}[?]{Ein wörtliches Zitat, das mit einem Fragezeichen endet}

\section{Wörtliche Zitate}
\subsection{Satz endet mit Zitat}
\zit[Im Vorwort zu][]{Scherz}[]{Das haben Sie gesagt}.
\zit[im Vorwort zu][]{Scherz}[?]{Was glauben Sie}
\subsection{Satz wird nach dem Zitat fortgesetzt}
\zit[im Vorwort zu][]{Scherz}[?]{Was glauben Sie}, könnte man dazu fragen.
 
\Textcite[123]{Lessing} erkannte bereits: "`Ein Tisch ist kein gutes Bett."'
\fi

\section{Händisch gesetzte Beispiele}
\subsection{Direkte Zitate}
\textsc{Niemand} (1983, S. 123) erkannte bereits: "`Ein Tisch ist kein
gutes Bett."' Dennoch dauerte es Jahrzehnte, bis diese Erkenntnis in der
Fachwelt gebührende Anerkennung fand. "`Das ist alles nur eine Frage der
Matratze"', entgegnete zum Beispiel \textsc{Autor} (Jahr, S. 12).

\textsc{Niemand}\footnote{\textsc{Niemand}, \textit{Nichts}, S. 123}
erkannte bereits: "`Ein Tisch ist kein gutes Bett."'  Dennoch dauerte es
Jahrzehnte, bis diese Erkenntnis in der Fachwelt gebührende Anerkennung
fand. "`Das ist alles nur eine Frage der Matratze"', entgegnete zum
Beispiel \textsc{Autor}.\footnote{\textsc{Autor}, \textit{Titel}, S. 12}

"`Eine Waschrumpel ist kein Federbett."' (\textsc{Autor}, Jahr, S. 123)
"`Eine Waschrumpel ist kein Federbett."'\footnote{\textsc{Autor},
\textit{Titel}, S. 123}

"`Eine Waschrumpel ist kein Federbett"' (\textsc{Autor}, Jahr, S. 123), könnte man meinen.
"`Eine Waschrumpel ist kein Federbett"'\footnote{\textsc{Autor},
\textit{Titel}, S. 123}, könnte man meinen.

% TODO: Eigentlich besser nur links einziehen.
\begin{addmargin}[1cm]{0pt}
\itshape\strut\hbox to 0pt{\hss"`}Blockzitat mit Anführungszeichen. Eine Waschrumpel
ist kein Federbett. Bekanntlich leidet aber selbst bei einem Federbett
die Schlafqualität erheblich unter einer einzelnen Erbse, die unter der
Matratze platziert wird."' (Autor, Jahr, S. 123)
\end{addmargin}

% Eigentlich besser nur links einziehen.
\begin{addmargin}[1cm]{0pt}
\strut\hbox to 0pt{\hss"`}Blockzitat mit Anführungszeichen. Eine Waschrumpel
ist kein Federbett. Bekanntlich leidet aber selbst bei einem Federbett
die Schlafqualität erheblich unter einer einzelnen Erbse, die unter der
Matratze platziert wird."'\footnote{\textsc{Autor}, \textit{Titel}, S.
123}
\end{addmargin}

\subsection{Indirekte Zitate}
% TODO: es gibt unterschiedliche Ansichten darüber, ob der Fußnotentext
% mit einem Großbuchstaben beginnen soll.
\textsc{Autor} (Jahr, S.\@ 123) meint, dass eine Waschrumpel kein
Federbett sei.
\textsc{Autor}\footnote{Vgl.\@ \textsc{Autor}, \textit{Titel}, S. 123}
meint, dass eine Waschrumpel kein Federbett sei.

Eine Waschrumpel ist bekanntlich kein Federbett. (Vgl.\@ \textsc{Autor},
Jahr, S. 123) Eine Waschrumpel ist bekanntlich kein
Federbett.\footnote{Vgl.\@ \textsc{Autor}, \textit{Titel}, S. 123}

% \KOMAoptions{footnotes=multiple}
Eine Waschrumpel ist zwar kein Federbett (vgl.\@ \textsc{Autor}, Jahr,
S. 123), aber dennoch ist sie ein gutes Musikinstrument. Eine
Waschrumpel ist zwar kein Federbett\footnote{Vgl.\@ \textsc{Autor},
\textit{Titel}, S.
123}\multiplefootnoteseparator\footnote{Experimentelle Fußnote mit ganz
viel Text. So viel Text, dass er nicht in einer Zeile Platz hat. Wegen
der kleinen Schrift in der Fußnote, muss das ganz schön viel Text
sein.},
 aber dennoch ist sie ein gutes Musikinstrument.

\section{Testfeld}
Im folgenden werden die Zitiermakros angewendet und sollten der
eingestellten Zitierweise entsprechend korrekt arbeiten.
\subsection{Direkte Zitate}
\Textcite[123]{Niemand} erkannte bereits: \blockquote{Ein Tisch ist kein
gutes Bett.} Dennoch dauerte es Jahrzehnte, bis diese Erkenntnis in der
Fachwelt gebührende Anerkennung fand. \blockquote{Das ist alles nur eine
Frage der Matratze}, entgegnete zum Beispiel \textcite[12]{Autor}.

\zit[123]{Autor}[.]{Eine Waschrumpel ist kein Federbett}
\zit[123]{Autor}{Eine Waschrumpel ist kein Federbett}, könnte man meinen.
\zit[123]{Autor}[.]{Blockzitat mit Anführungszeichen. Eine Waschrumpel
ist kein Federbett. Bekanntlich leidet aber selbst bei einem Federbett
die Schlafqualität erheblich unter einer einzelnen Erbse, die unter der
Matratze platziert wird}

\subsection{Indirekte Zitate}
% TODO: es gibt unterschiedliche Ansichten darüber, ob der Fußnotentext
% mit einem Großbuchstaben beginnen soll.
\Textcite[\bibstring{confer}][123]{Autor} meint, dass eine Waschrumpel kein
Federbett sei.

Eine Waschrumpel ist bekanntlich kein Federbett. \Vgl[123]{Autor} Eine
Waschrumpel ist zwar kein Federbett \vgl[123]{Autor}{}, aber dennoch ist
sie ein gutes Musikinstrument.


% Das Literaturverzeichnis
% !TeX root = document.tex
% 20180308T1028 Leonard Michlmayr

%% Einige Filter für die Einträge im Literaturverzeichnis
\defbibfilter{online}{( type=online or subtype=online )}
\defbibfilter{interview}{type=interview or subtype=interview}
\defbibfilter{onlinetext}{( type=online or subtype=online and not ( type=video
  or type=audio ) )}
\defbibfilter{offline}{not ( type=online or subtype=online )}
\defbibfilter{print}{not ( type=online or subtype=online or type=video or
  type=audio or type=interview or subtype=interview )}
\defbibfilter{offlinevideo}{type=video and not subtype=online}
\defbibfilter{offlineaudio}{type=audio and not subtype=online}
\defbibfilter{nurAusSekundaerliteratur}{category=quotee and not category=primary}
\defbibfilter{nichtNurAusSekundaerliteratur}{%
  category=quoter or category=primary or category=nocited}

% TODO: die Untergliederung des Literaturverzeichnisses den eigenen
% Bedürfnissen anpassen.
\printbibheading[heading=bibintoc,title=Literaturverzeichnis]\label{Lit}
\printshorthands[heading=subbibintoc]
\printbibliography[heading=subbibintoc,title={Print-Quellen},category=inbib,filter=print,filter=nichtNurAusSekundaerliteratur]
\printbibliography[heading=subbibintoc,title={Audio-Quellen},category=inbib,filter=offlineaudio,filter=nichtNurAusSekundaerliteratur]
\printbibliography[heading=subbibintoc,title={Video-Quellen},category=inbib,filter=offlinevideo,filter=nichtNurAusSekundaerliteratur]
\printbibliography[heading=subbibintoc,title={Internet-Quellen},category=inbib,filter=online,filter=nichtNurAusSekundaerliteratur]
\printbibliography[heading=subbibintoc,title={Sekundärzitate},category=inbib,filter=nurAusSekundaerliteratur]
\printbibliography[heading=subbibintoc,title={Interviews},category=inbib,filter=interview]


\listoffigures
% TODO: Wer keine Tabellen hat, muss das Tabellenverzeichnis entfernen!
% Bei kurzen Tabellenverzeichnissen kann man vielleicht
% Abbildungsverzeichnis und Tabellenverzeichnis auf einer Seite platzieren.
% \withoutclearpage unterdrückt die neue Seite.
\withoutclearpage{\listoftables}

% Gegebenenfalls ein Anhang
\appendix
% !TeX root = document.tex
\chapter{Interview mit Ex Perte}\label{InterviewMitExPerte}
\begin{flushright}
Interviewdatum: 25.\,Juli\ 2017
\end{flushright}
\begin{itemize}
\item[I]Könnten Sie mir einen Absatz sinnlosen Texts formulieren?

\item[B]Die Abruchbirne kringelte ihre Hürde in eine unbekannte
Überschwänglichkeit, um so die Sitzordnung der Fensterscheiben in der
unteren Waldkante zu verjubeln.
Niemandem ist absichtlich zu kürzen, wessen Woligkeit hier in
abermaligem Abgesang aufgeschlagen ist. Deswegen soll dieser aber nicht
heimreisen, sondern abermals die Einigkeit des Urwalds in die aufgeregte
Höhensonne schlagen.
Wenn nicht der hiesige Erdball des aberwitzigen Ungemachs aufgedrungene
Kröte wäre, entschließe ich mich zu unsachgemäßem Handlungsablauf.
Wiegleich zudem ein weiterer Honigkuchen ausbricht.

\item[I] Könnten Sie mir einen Absatz sinnlosen Texts formulieren?

\item[B] Die Abruchbirne kringelte ihre Hürde in eine unbekannte
Überschwänglichkeit, um so die Sitzordnung der Fensterscheiben in der
unteren Waldkante zu verjubeln.
Niemandem ist absichtlich zu kürzen, wessen Woligkeit hier in
abermaligem Abgesang aufgeschlagen ist. Deswegen soll dieser aber nicht
heimreisen, sondern abermals die Einigkeit des Urwalds in die aufgeregte
Höhensonne schlagen.
Wenn nicht der hiesige Erdball des aberwitzigen Ungemachs aufgedrungene
Kröte wäre, entschließe ich mich zu unsachgemäßem Handlungsablauf.
Wiegleich zudem ein weiterer Honigkuchen ausbricht.

\item[I] Könnten Sie mir einen Absatz sinnlosen Texts formulieren?

\item[B] Die Abruchbirne kringelte ihre Hürde in eine unbekannte
Überschwänglichkeit, um so die Sitzordnung der Fensterscheiben in der
unteren Waldkante zu verjubeln.
Niemandem ist absichtlich zu kürzen, wessen Woligkeit hier in
abermaligem Abgesang aufgeschlagen ist. Deswegen soll dieser aber nicht
heimreisen, sondern abermals die Einigkeit des Urwalds in die aufgeregte
Höhensonne schlagen.\label{interessanteStelleImInterview}
% Hier wird eine interessante Stelle im Interview markiert um darauf
% referenzieren zu können.
Wenn nicht der hiesige Erdball des aberwitzigen Ungemachs aufgedrungene
Kröte wäre, entschließe ich mich zu unsachgemäßem Handlungsablauf.
Wiegleich zudem ein weiterer Honigkuchen ausbricht.

\item[I] Könnten Sie mir einen Absatz sinnlosen Texts formulieren?

\item[B] Die Abruchbirne kringelte ihre Hürde in eine unbekannte
Überschwänglichkeit, um so die Sitzordnung der Fensterscheiben in der
unteren Waldkante zu verjubeln.
Niemandem ist absichtlich zu kürzen, wessen Woligkeit hier in
abermaligem Abgesang aufgeschlagen ist. Deswegen soll dieser aber nicht
heimreisen, sondern abermals die Einigkeit des Urwalds in die aufgeregte
Höhensonne schlagen.
Wenn nicht der hiesige Erdball des aberwitzigen Ungemachs aufgedrungene
Kröte wäre, entschließe ich mich zu unsachgemäßem Handlungsablauf.
Wiegleich zudem ein weiterer Honigkuchen ausbricht.
\end{itemize}
% !TeX root = document.tex
\chapter{Hier könnte Ihr Anhang stehen}


\backmatter

%\pdfbookmark[0]{Erklärungen}{erkl}
\addchap{Erklärungen}
\section*{Selbstständigkeitserklärung}
\thispagestyle{plain}
Ich erkläre, dass ich diese vorwissenschaftliche Arbeit eigenständig
angefertigt und nur die im Literaturverzeichnis angeführten Quellen und
Hilfsmittel benutzt habe.

\vspace{2cm}\noindent Wien, \today

% TODO: Erkläre dich selbstständig selbstständig! 
\vspace{2cm}\noindent\makeatletter\@AutorIn\makeatother

\vspace{2cm}\noindent

\section*{Informatikschwerpunkt}

Die vorliegende Arbeit erfüllt die Kriterien zur Abbildung des
Informatikschwerpunktes an der De La Salle Schule Strebersdorf, AHS.

\textbf{Begründung:} Die Arbeit wurde in \LaTeX{} mit entscheidenden 
Kenntnissen zum Quelltext verfasst.\vspace{.5\baselineskip}

\noindent\textit{Geprüft am \ldots durch Mag. Rainer Zufall und Mag.
Ernst Haft}

\end{document}
